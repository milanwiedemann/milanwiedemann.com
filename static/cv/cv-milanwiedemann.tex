%----------------------------------------------------------------------------------------
%	PACKAGES AND OTHER DOCUMENT CONFIGURATIONS
%----------------------------------------------------------------------------------------

\documentclass{resume} % Use the custom resume.cls style
% \pagestyle{plain} % uncomment for page numbers
\usepackage[left = 0.75in, top = 0.6in, right = 0.75in, bottom = 0.6in]{geometry} % Document margins
\usepackage[small]{titlesec} % To change font of subheadings for publications
\usepackage[T1]{fontenc} % currency signs pound
\usepackage{eurosym} % currency signs euro
\usepackage{hyperref} % hyperref for email and website

% for numbered, max bib names in case you are way down the list of authors
% check out bib files to see how to 
% I like how nature looks!
%\usepackage[backend = biber, sorting = none, maxbibnames = 99, style = nature]{biblatex}

% for apa6
\usepackage[style = apa, backend = biber, language = american]{biblatex}
\DeclareLanguageMapping{english}{american-apa}

\addbibresource{publications.bib}
\addbibresource{presentations.bib}

\name{Milan Wiedemann}
\address{Wolfson College, Linton Road, OX2 6UD Oxford} % Your address
%\address{\href{mailto:milan.wiedemann@gmail.com}{milan.wiedemann@gmail.com}}

% To add website uncomment this and comment out the above line
\address{\href{mailto:milan.wiedemann@gmail.com}{milan.wiedemann@gmail.com} \\ \href{http://www.milanwiedemann.com/}{www.milanwiedemann.com}} 


% Add option to have item-less subsections
% Code from https://tex.stackexchange.com/questions/370390/in-the-resume-document-class-how-to-create-a-subsection-without-items/370429

\newenvironment{rEmptySubsection}[4]{
  %%%%%%%%%%%%%%%%%%%%%% Default Layout: %%%%%%%%%%%%%%%%%%%%%%%%
  %%    Employer (bold)                     Dates (regular)    %%
  %%    Title (emphasis)                Location (emphasis)    %%
  %%%%%%%%%%%%%%%%%%%%%%%%%%%%%%%%%%%%%%%%%%%%%%%%%%%%%%%%%%%%%%%
  {\bf #1}                 \hfill                  {    #2}% Stop a space
  \ifthenelse{\equal{#3}{}}{}{
  \\
  {\em #3}                 \hfill                  {\em #4}% Stop a space
  }
  % empty
}{
}

\begin{document}

%----------------------------------------------------------------------------------------
%	PERSONAL SUMMARY
%----------------------------------------------------------------------------------------

\begin{rSection}{Personal Summary}
I am a PhD candidate at the University of Oxford investigating processes of change in cognitive therapy for Posttraumatic Stress Disorder (PTSD).
Testing the theoretical models underpinning cognitive therapy for PTSD has strengthened my understanding of the importance that clinical interventions are rooted in empirical science and
continually evaluated and refined through research. 
I appreciate the importance of innovative research projects testing and evaluating technological advances in clinical psychology.
I value, advocate, and implement open science practices in my work.
\end{rSection}

%----------------------------------------------------------------------------------------
%	EDUCATION SECTION
%----------------------------------------------------------------------------------------

\begin{rSection}{Education and Qualifications}
{\bf University of Oxford} \hfill {\em October 2016 - Present} \\ 
DPhil in Experimental Psychology at the Oxford Centre for Anxiety Disorders and Trauma \smallskip \\
Thesis: Processes of Change in Cognitive Therapy for Posttraumatic Stress Disorder \\
Supervised by Prof. Anke Ehlers and Prof. David M. Clark \smallskip

{\bf Technische Universit{\"a}t Braunschweig} \hfill {\ August 2013 - September 2015} \\ 
MSc in Psychology \\
Study  focus: Clinical Psychology \smallskip

{\bf Technische Universit{\"a}t Braunschweig} \hfill {\ August 2011 - September 2013} \\ 
BSc in Psychology \smallskip

{\bf University of Groningen} \hfill {\ September 2010 - August 2011} \\ 
BSc in Psychology \smallskip

{\bf DRK-Rettungsschule Goslar} \hfill {\ Janurary 2011- February 2011} \\ 
Rettungssanit{\"a}ter (in English `Paramedic') \smallskip

\end{rSection}

%----------------------------------------------------------------------------------------
%	RESEARCH EXPERIENCE
%----------------------------------------------------------------------------------------

\begin{rSection}{Clinical \& Research Experience}

\begin{rSubsection}{University of Oxford}{\em October 2016 - Present}{Part-time Research Assistant, Oxford Centre for Anxiety Disorders and Trauma}{Oxford, UK}
\item Creating videos for training clinical psychologists in specialised cognitive behavioural treatments.
\item Teaching and supervising Research Assistants in filming and editing techniques.
\end{rSubsection}

\begin{rSubsection}{University of Oxford}{April 2015 - August 2016}{Research Assistant, Oxford Centre for Anxiety Disorders and Trauma}{Oxford, UK}
\item Production of patient testimonials and educational videos together with clinical psychologists.
\item Animation of psycho-educational illustrations.
\item Administration of text, video and audio content for online therapy programmes.
\end{rSubsection}

\begin{rSubsection}{Technische Universit{\"a}t Braunschweig}{September 2012 - March 2015}{Part-time Research Assistant, Institute of Clinical Psychology}{Braunschweig, GER}
\item Conducting clinical interviews with adolescents for a longitudinal study.
\item Setting up and managing large databases for national and international research teams.
\end{rSubsection}

\begin{rSubsection}{Local Health Authority Berlin Pankow}{February 2012 - March 2012}{Intern, Social Psychiatric Department}{Berlin, GER}
\item Conducting initial clinical interviews as well as follow-up interviews under supervision.
\item Making house calls as part of a multidisciplinary team in emergency situations.

\end{rSubsection}

\end{rSection}

%----------------------------------------------------------------------------------------
%	TEACHING EXPERIENCE
%----------------------------------------------------------------------------------------

% Add how many hours per week 

\begin{rSection}{Teaching Experience}

\begin{rSubsection}{University of Oxford}{January 2018 - February 2018}{Demonstrator, Department of Experimental Psychology}{Oxford, UK}
\item Module: Psychological Disorders and Individual Differences.
\end{rSubsection}

\begin{rSubsection}{Technische Universit{\"a}t Braunschweig}{April 2014 - March 2015}{Teaching Assistant, Institute of Clinical Psychology}{Braunschweig, GER}
\item Module: Practical Course of Psychological Diagnostics.
\item Module: Fundamentals of Psychological Diagnostics.
\end{rSubsection}

\begin{rSubsection}{Technische Universit{\"a}t Braunschweig}{September 2012 - January 2014}{Teaching Assistant, Institute of Methodology and Biopsychology}{Braunschweig, GER}
\item Module: Laboratory Experience and Analysis with Statistical Software.
\end{rSubsection}

\end{rSection}

%----------------------------------------------------------------------------------------
%	ADDITIONAL EXPERIENCE
%----------------------------------------------------------------------------------------

\begin{rSection}{Additional Experience}

\begin{rSubsection}{University of Oxford}{July 2017 - September 2017}{Part-time Research Assistant, Faculty of History}{Oxford, UK}
\item Production of promotional video for the workshop: Changing lives: Understanding what enables children growing up in adversity to thrive in modern Britain.
\end{rSubsection}

\begin{rSubsection}{Arbeiter-Samariter-Bund Braunschweig}{May 2010 - March 2015}{Part-time Paramedic}{Braunschweig, GER}
\item Medical care during the qualified transport of patients.
\item First aid and emergency treatment of patients.
\item Independent ambulance coach driver in the in-house emergency centre.
\end{rSubsection}

\begin{rSubsection}{Arbeiter-Samariter-Bund Braunschweig}{July 2008 - March 2009}{Civilian Service}{Braunschweig, GER}
\item Getting trained as a `Paramedic' (in German: Rettungssanit{\"a}ter).
\item Medical care during the qualified transport of patients.
\item Independent ambulance coach driver in the in-house emergency centre.
\end{rSubsection}

\end{rSection}

%----------------------------------------------------------------------------------------
%	Scholarships
%----------------------------------------------------------------------------------------

\begin{rSection}{Scholarships}

\begin{rEmptySubsection}{Mental Health Research UK}{May 2018}{Travel Grant}{}
\end{rEmptySubsection}

\begin{rEmptySubsection}{University of Oxford, Wolfson College}{June 2017}{Travel Grant}{}
\end{rEmptySubsection}

\begin{rEmptySubsection}{Mental Health Research UK}{October 2016}{PhD Scholarship}{}
\end{rEmptySubsection}

\begin{rEmptySubsection}{Erasmus plus}{February 2015}{Graduate Student Scholarship}{}
\end{rEmptySubsection}

\end{rSection}

%----------------------------------------------------------------------------------------
%	Professional development
%----------------------------------------------------------------------------------------

\begin{rSection}{Selected Professional Development}

\begin{rEmptySubsection}{DeGPT Autumn School}{October 2018}{Deutschsprachige Gesellschaft f{\"u}r Psychotraumatologie, 4 days}{} 
\end{rEmptySubsection}

\begin{rEmptySubsection}{Psychological Networks Summer School}{August 2018}{University of Amsterdam, 5 days}{} 
\end{rEmptySubsection}

\begin{rEmptySubsection}{Summer School of Advanced Quantitative Methods}{May 2018}{University of Oxford, Department of Education, 5 days}{} 
\end{rEmptySubsection}

\begin{rEmptySubsection}{Testing for Mediation and Moderation Using Mplus}{June 2017}{Figure It Out - Statistical Consultancy and Training, 1 day}{}
\end{rEmptySubsection}

\begin{rEmptySubsection}{Multilevel Analysis for Longitudinal Data}{February 2016}{Universit{\"a}t Landau, Methodenzentrum, 3 days}{}
\end{rEmptySubsection}

\begin{rEmptySubsection}{Structural Equation Models for Longitudinal Data}{February 2016}{Universit{\"a}t Landau, Methodenzentrum, 2 days}{}
\end{rEmptySubsection}

\end{rSection}

%----------------------------------------------------------------------------------------
%	TECHNICAL STRENGTHS SECTION
%----------------------------------------------------------------------------------------

\begin{rSection}{Technical Strengths}

\begin{tabular}{ @{} >{\bfseries}l @{\hspace{6ex}} l }
Statistics \& Data science & R, Mplus, JASP, jamovi, SPSS, GitHub \\
Scientific writing & R Markdown, Zotero, LaTeX, BibTeX, Microsoft Office \\
Video \& Audio editing & Final Cut X, Audacity, iZotope RX \\
%Driving licence &Clean category B \\
\end{tabular}
\end{rSection}

%----------------------------------------------------------------------------------------
% Publications & Presentations
%----------------------------------------------------------------------------------------

\begin{rSection}{Publications}

% Peer-reviewed
\begin{refsection}
\printbibliography[keyword={peer-reviewed},title={Peer-reviewed}]

\nocite{Wild2016}

\end{refsection}

% Preprint
\begin{refsection}
\printbibliography[keyword={preprint}, title={Preprint}]

% Add preprints here

\end{refsection}

% Software
\begin{refsection}
\printbibliography[keyword={software},title={Software}]

\nocite{R-suddengains}
\nocite{R-lcsm}

\end{refsection}
\end{rSection}

% Presentations
\begin{rSection}{Presentations}

% Conference presentations
\begin{refsection}
\printbibliography[keyword={conference-presentation}, title={Conference presentations}]

\nocite{Wiedemann2017-EABCT}

\end{refsection}

% Invited talks
\begin{refsection}
\printbibliography[keyword={invited-presentation}, title={Invited talks}]

\nocite{Wiedemann2018-Groningen}
\nocite{Wiedemann2018-Exeter}
\nocite{Wiedemann2016-Berlin}
\nocite{Wiedemann2015-Berlin}

\end{refsection}
\end{rSection}

%----------------------------------------------------------------------------------------

\end{document}
